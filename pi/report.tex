\documentclass[10pt,a4paper]{article}
\usepackage[utf8]{inputenc}
\usepackage{amsmath}
\usepackage{amsfonts}
\usepackage{amssymb}
\usepackage{fullpage}
\author{Patrick White}
\begin{document}
\subsection{Usage}
To use the inter-procedural analysis feature, add the expansion depth as a fourth command line parameter to the program: \texttt{./pipair 3 65} will not expand nodes. \texttt{./pipair 3 65 1} will expand nodes one level deep. \texttt{./pipair 3 65 0} will expand to an infinite depth.
\subsection{Solution Description}
When using inter-procedural analysis, the program will expand nodes up to the depth provided, so that each node keeps track of functions that are actually called by the node, and also a set of functions that are called up to the depth provided. This does not affect the support values of the functions. Consider the following example with $depth=1$:
\begin{verbatim}
void A(){}
void B(){}
void C(){
	B();
}
void D(){
	A();
	C();
}
\end{verbatim}
In this example, $support(\{A,B\})$ is still $0$ because $A$ and $B$ are never called together in the same function, even though $D$ becomes expanded so that its expanded call set is $\{A,B,C\}$. If they did affect each other's $support$ levels, $C$ would be considered to be a bug, since $D$
\subsection{Experiment}
\subsection{Validity of Solution}
\end{document}